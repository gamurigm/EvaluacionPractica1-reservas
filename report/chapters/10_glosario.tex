\section{Glosario}

\begin{description}
    \item[API REST] Interfaz de programación que utiliza protocolos HTTP estándar para operaciones de gestión de recursos (CRUD).
    
    \item[Code Smell] Indicador en el código fuente que sugiere problemas de diseño o mantenibilidad, aunque no sea técnicamente un error.
    
    \item[Cobertura de Pruebas] Métrica que indica el porcentaje de líneas, ramas o caminos del código fuente que son ejecutados durante las pruebas automatizadas.
    
    \item[CRUD] Acrónimo de las operaciones básicas de almacenamiento persistente: Create (Crear), Read (Leer), Update (Actualizar), Delete (Eliminar).
    
    \item[Deuda Técnica] Costo implícito de retrabajo adicional causado por elegir una solución fácil/rápida en lugar de una mejor solución que tomaría más tiempo.
    
    \item[Fuzzing] Técnica de pruebas de software que implica proporcionar datos inválidos, inesperados o aleatorios a las entradas de un programa.
    
    \item[JWT (JSON Web Token)] Estándar abierto para la creación de tokens de acceso que permiten la propagación de identidad y claims entre partes.
    
    \item[Throughput] Medida de cuántas unidades de información (solicitudes, transacciones) puede procesar un sistema en un periodo de tiempo dado.

    \item[Quality Gate] Umbral de calidad definido en SonarQube que un proyecto debe cumplir obligatoriamente para ser considerado apto para producción.

    \item[Ramp-up] Periodo de tiempo configurado en JMeter para que el sistema alcance gradualmente el número total de usuarios concurrentes definidos.

    \item[Endpoint] Punto de acceso o URL específica de una API que representa un recurso y permite interactuar con él mediante métodos HTTP.

    \item[Análisis Estático] Técnica de evaluación de software que se realiza sin ejecutar el código, buscando patrones de error o malas prácticas.

    \item[Pruebas Unitarias] Pruebas de bajo nivel que validan el funcionamiento correcto de componentes o funciones individuales de forma aislada.

    \item[Pruebas de Integración] Verificación de que diferentes módulos o servicios del sistema interactúan correctamente entre sí.

    \item[Pruebas de Regresión] Re-ejecución de pruebas para asegurar que los nuevos cambios o correcciones no han introducido errores en funcionalidades existentes.

    \item[CI/CD] Integración Continua y Entrega Continua; práctica de automatizar el ciclo de vida de desarrollo, pruebas y despliegue.

    \item[JMX] Formato de archivo de plan de pruebas utilizado por Apache JMeter para almacenar la configuración de los escenarios de carga.
\end{description}
