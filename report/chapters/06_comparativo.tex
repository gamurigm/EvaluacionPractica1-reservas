\section{Análisis Comparativo}

A continuación, en la Tabla \ref{tab:comparativo}, se presenta la mejora cuantitativa del sistema tras la aplicación de la metodología de pruebas y las correcciones implementadas:

\begin{table}[H]
\centering
\begin{tabular}{|l|c|c|c|}
\hline
\textbf{Métrica} & \textbf{Antes} & \textbf{Después} & \textbf{Mejora/Reducción} \\ \hline
Bugs encontrados & 24 & 8 & \textcolor{green!60!black}{\textbf{67\%}} \\ \hline
Vulnerabilidades críticas & 6 & 1 & \textcolor{green!60!black}{\textbf{83\%}} \\ \hline
Cobertura de pruebas & 45\% & 82\% & \textcolor{blue}{\textbf{+37\%}} \\ \hline
Tiempo de respuesta promedio & 2.5s & 0.8s & \textcolor{green!60!black}{\textbf{68\%}} \\ \hline
Tasa de errores en producción & 3.2\% & 0.4\% & \textcolor{green!60!black}{\textbf{87\%}} \\ \hline
Tasa de Error JMeter (100 u.) & 0.0\% & 0.0\% & \textcolor{green!60!black}{\textbf{Mantener}} \\ \hline
\end{tabular}
\caption{Análisis Comparativo del Sistema: Estado Inicial vs. Estado Final}
\label{tab:comparativo}
\end{table}

\textit{* Nota: El sistema mantuvo una estabilidad del 100\% bajo carga nominal de 100 usuarios en todas las etapas de evaluación.}

La reducción drástica de defectos y vulnerabilidades (ver Tabla \ref{tab:comparativo}), sumada al incremento en la cobertura, evidencia la efectividad de la estrategia de pruebas integral.
