\section{Introducción}

\subsection{Contexto del Proyecto}
El proyecto objeto de evaluación es un sistema integral de gestión de reservas y turnos diseñado para entornos de producción. Este sistema implementa una arquitectura moderna que incluye:
\begin{itemize}
    \item \textbf{Backend:} Implementado mediante arquitectura REST con endpoints escalables.
    \item \textbf{Autenticación:} Mecanismo de autenticación y autorización basado en tokens JWT (JSON Web Tokens), incluyendo soporte para claims de roles y refresh tokens.
    \item \textbf{Persistencia:} Capa de datos con soporte para transacciones ACID, encargada de persistir reservas, usuarios y registros de auditoría.
    \item \textbf{Funcionalidad:} Operaciones CRUD (Create, Read, Update, Delete) completamente funcionales.
\end{itemize}

El entorno de pruebas cuenta con usuarios activos y datos representativos, constituyendo un caso de estudio idóneo para la aplicación sistemática de técnicas de aseguramiento de la calidad de software (QA).

\subsection{Objetivo General del Laboratorio}
El objetivo principal de este laboratorio es aplicar un conjunto integral de técnicas de pruebas de software mediante un enfoque metodológico estructurado. Esto incluye la integración de:
\begin{itemize}
    \item Pruebas estáticas
    \item Pruebas funcionales dinámicas
    \item Pruebas unitarias y de integración
    \item Pruebas de sistema
    \item Pruebas de seguridad
    \item Pruebas de rendimiento y carga
    \item Automatización de pruebas
\end{itemize}
Todo ello con la finalidad de validar la calidad, robustez y seguridad del proyecto web en un escenario de producción simulado.
