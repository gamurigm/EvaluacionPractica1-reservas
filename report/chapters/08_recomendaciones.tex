\section{Recomendaciones}

Basado en los hallazgos y los retos técnicos superados durante este laboratorio, se plantean las siguientes recomendaciones de carácter técnico y metodológico:

\subsection{Lecciones Aprendidas y Soluciones Aplicadas}
Durante la ejecución de las pruebas se identificaron varios cuellos de botella técnicos que fueron resueltos mediante las siguientes estrategias:

\begin{itemize}
    \item \textbf{Configuración de OWASP ZAP:} Se recomienda asegurar que el servicio objetivo (Target URL) esté activo y sea accesible desde el entorno donde corre el escáner. Se solucionaron errores de conexión especificando explícitamente el puerto del servidor local y verificando que no existieran firewalls bloqueando el tráfico de escaneo activo.
    
    \item \textbf{Gestión de Errores en JMeter (Connection Refused):} Ante los fallos observados con 1000 usuarios, se recomienda optimizar la configuración del pool de conexiones del servidor y la base de datos. Se solucionó la interpretación de estos errores identificando el punto de saturación del hardware local (bottleneck) y ajustando el tiempo de ramp-up en el JMX.
    
    \item \textbf{Ejecución Automatizada con Newman:} Para evitar errores de rutas (\textit{ENOENT}), se recomienda estandarizar la ejecución de comandos siempre desde el directorio raíz del proyecto. Esto garantiza que las colecciones de Postman y las referencias a los entornos sean consistentes.
\end{itemize}

\subsection{Mejora Continua del Sistema}
\begin{enumerate}
    \item \textbf{Integración Continua:} Implementar de forma definitiva el pipeline de \textbf{GitHub Actions} para ejecutar la suite de pruebas automatizadas y el análisis de calidad en cada commit, asegurando que no existan regresiones en producción.
    
    \item \textbf{Auditorías de Seguridad:} Realizar auditorías de seguridad trimestrales utilizando \textbf{OWASP ZAP} para detectar nuevas vulnerabilidades surgidas por actualizaciones de dependencias o cambios en la lógica de negocio.
    
    \item \textbf{Métricas de Calidad:} Mantener el umbral de \textbf{cobertura mínima del 80\%} y la densidad de defectos por debajo de 1.0/KLOC como requisitos de aceptación para cualquier nueva funcionalidad.
    
    \item \textbf{Pruebas de Estrés:} Utilizar entornos de Staging con especificaciones similares a producción para validaciones de carga superiores a 500 usuarios, evitando así interferencias por limitaciones de hardware local.
\end{enumerate}
