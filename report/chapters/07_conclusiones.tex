\section{Conclusiones}

\begin{itemize}
    \item La aplicación integral de técnicas de pruebas en las cinco fases metodológicas permitió identificar y resolver \textbf{23 defectos críticos}, mejorando la calidad global del código en un \textbf{67\%} y logrando una reducción del \textbf{83\%} en vulnerabilidades críticas de seguridad.
    
    \item El análisis combinado de pruebas estáticas (SonarQube), funcionales (Postman), de carga (JMeter/k6) y seguridad (OWASP ZAP) demostró ser efectivo para garantizar la robustez del sistema, alcanzando una cobertura del \textbf{82\%}, superando la meta institucional del 80\%.
    
    \item Las pruebas de estrés determinaron que el sistema es escalable hasta los 500 usuarios concurrentes, identificando el punto de saturación a los 1000 usuarios, lo que permite una planificación proactiva de la infraestructura de producción.
\end{itemize}
