\section{Alcance}

Este informe abarca la evaluación completa del ciclo de vida de pruebas del sistema de gestión de reservas, cubriendo los siguientes componentes y actividades:

\subsection{Historias de Usuario Evaluadas}
De acuerdo con las especificaciones del proyecto (Slide 4), se validaron los siguientes escenarios de usuario:
\begin{itemize}
    \item \textbf{HU-01:} Como usuario, quiero crear una reserva para asegurar mi lugar en el sistema.
    \item \textbf{HU-02:} Como usuario, quiero consultar mis reservas para verificar los detalles de mis turnos.
    \item \textbf{HU-03:} Como administrador, quiero gestionar la disponibilidad para optimizar el servicio.
\end{itemize}

\begin{itemize}
    \item \textbf{Componentes Evaluados:} 
    \begin{itemize}
        \item API REST y lógica de negocio.
        \item Base de datos y mecanismos de persistencia.
        \item Módulos de autenticación y seguridad.
        \item Interfaces de gestión (simuladas mediante llamadas API).
    \end{itemize}
    
    \item \textbf{Actividades Incluidas:}
    \begin{itemize}
        \item Análisis estático de código fuente.
        \item Ejecución manual y automatizada de pruebas funcionales.
        \item Evaluación de rendimiento bajo condiciones de carga normal y estrés.
        \item Análisis de vulnerabilidades de seguridad web.
        \item Implementación de pipelines de Integración Continua (CI/CD).
    \end{itemize}
    
    \item \textbf{Exclusiones:}
    \begin{itemize}
        \item No se incluye el desarrollo de nuevas funcionalidades, únicamente la validación de las existentes.
        \item Pruebas de interfaz de usuario (UI) a nivel gráfico profundo (se prioriza la validación lógica y de backend).
    \end{itemize}
\end{itemize}

\subsection{Matriz de Trazabilidad: Requisitos vs Pruebas}
Para asegurar una cobertura integral, se ha definido la siguiente relación entre los requisitos del sistema y las fases de prueba:

\begin{table}[H]
\centering
\begin{tabular}{|l|p{8cm}|}
\hline
\textbf{Requisito} & \textbf{Cubierto por} \\ \hline
REQ-01: Crear reserva & Pruebas Funcionales, Pruebas de Carga, Pruebas de Seguridad \\ \hline
REQ-02: Autenticación & Pruebas de Seguridad, Pruebas Funcionales \\ \hline
REQ-03: Validación de fechas & Pruebas Unitarias, Pruebas Funcionales \\ \hline
REQ-04: Rendimiento < 1s & Pruebas de Carga, Pruebas de Rendimiento \\ \hline
REQ-05: Disponibilidad 99.9\% & Pruebas de Recuperación, Pruebas de Estrés \\ \hline
\end{tabular}
\caption{Matriz de trazabilidad de requisitos y pruebas}
\label{tab:trazabilidad}
\end{table}

\subsection{Cronograma de Ejecución}
El laboratorio se estructuró en un periodo de 5 semanas, distribuidas de la siguiente manera:

\begin{itemize}
    \item \textbf{Semana 1: Análisis Estático.} Ejecución con SonarQube y revisión manual.
    \item \textbf{Semana 2: Pruebas Funcionales.} Validación de endpoints con Postman/Newman.
    \item \textbf{Semana 3: Pruebas de Sistema.} Evaluación con JMeter y OWASP ZAP.
    \item \textbf{Semana 4: Pruebas de Carga.} Automatización de escenarios con k6.
    \item \textbf{Semana 5: Automatización y Documentación.} Integración CI/CD y reporte final.
\end{itemize}
