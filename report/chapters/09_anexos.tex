\section{Anexos}

\subsection{Evidencia Visual de Pruebas}

\begin{figure}[H]
    \centering
    \includegraphics[width=0.8\textwidth]{images/cap10-sonnar-all-passed.png}
    \caption{Dashboard General de SonarQube mostrando 0 hallazgos tras corrección}
    \label{fig:sonar_dash}
\end{figure}

\begin{figure}[H]
    \centering
    \includegraphics[width=0.8\textwidth]{images/postman0.png}
    \caption{Configuración inicial de la Colección de Pruebas en Postman}
    \label{fig:postman_gui}
\end{figure}

\begin{figure}[H]
    \centering
    \includegraphics[width=0.8\textwidth]{images/postman.png}
    \caption{Resultados de ejecución exitosa de Pruebas de Integración (Postman/Newman)}
    \label{fig:postman_res}
\end{figure}

\begin{figure}[H]
    \centering
    \includegraphics[width=0.8\textwidth]{images/jmeter-100.png}
    \caption{Resultados de Carga Nominal (100 usuarios) - 24.0\% de error}
    \label{fig:jmeter_100}
\end{figure}

\begin{figure}[H]
    \centering
    \includegraphics[width=0.8\textwidth]{images/jmeter-250.png}
    \caption{Resultados de Carga Media (250 usuarios)}
    \label{fig:jmeter_250}
\end{figure}

\begin{figure}[H]
    \centering
    \includegraphics[width=0.8\textwidth]{images/jmeter-500.png}
    \caption{Resultados de Carga de Estrés (500 usuarios)}
    \label{fig:jmeter_500}
\end{figure}

\begin{figure}[H]
    \centering
    \includegraphics[width=0.8\textwidth]{images/jmeter-1000.png}
    \caption{Resultados de Punto de Ruptura (1000 usuarios)}
    \label{fig:jmeter_1000}
\end{figure}

\begin{figure}[H]
    \centering
    \includegraphics[width=0.9\textwidth]{images/CICD.png}
    \caption{Historial de ejecuciones de CI/CD mostrando la transición al estado exitoso (Verde)}
    \label{fig:anexo_cicd}
\end{figure}

\subsection{Código de Pruebas Implementado}

A continuación se presenta el código fuente íntegro de los scripts de pruebas desarrollados para la validación del sistema en sus diferentes capas.

\subsubsection{Pruebas Funcionales y de Integración (tests/functional.test.js)}
\begin{lstlisting}[language=JavaScript, caption=Suite íntegra de pruebas funcionales y de seguridad con Jest]
const request = require('supertest');
const mongoose = require('mongoose');
const app = require('../src/app');
const User = require('../src/models/User');
const Reserva = require('../src/models/Reserva');

jest.setTimeout(30000);
let token;

afterAll(async () => {
    await mongoose.connection.close();
});

describe('Functional Tests - API Reservas', () => {
    beforeAll(async () => {
        await User.deleteMany({ email: 'test@example.com' });
        await Reserva.deleteMany({ nombreCliente: 'Test Client' });
    });

    describe('Authentication', () => {
        it('should register a new user', async () => {
            const res = await request(app)
                .post('/api/auth/register')
                .send({
                    username: 'testuser',
                    email: 'test@example.com',
                    password: 'password123'
                });
            expect(res.statusCode).toEqual(201);
            expect(res.body).toHaveProperty('msg');
        });

        it('should login with valid credentials', async () => {
            const res = await request(app)
                .post('/api/auth/login')
                .send({
                    email: 'test@example.com',
                    password: 'password123'
                });
            expect(res.statusCode).toEqual(200);
            expect(res.body).toHaveProperty('token');
            token = res.body.token;
        });

        it('should not login with invalid password', async () => {
            const res = await request(app)
                .post('/api/auth/login')
                .send({
                    email: 'test@example.com',
                    password: 'wrongpassword'
                });
            expect(res.statusCode).toEqual(400);
        });
    });

    describe('Reservations', () => {
        it('should not create reservation without token', async () => {
            const res = await request(app)
                .post('/api/reservas')
                .send({
                    fecha: '2025-12-01',
                    hora: '10:00',
                    nombreCliente: 'Test Client'
                });
            expect(res.statusCode).toEqual(401);
        });

        it('should create reservation with valid token', async () => {
            const res = await request(app)
                .post('/api/reservas')
                .set('Authorization', `Bearer ${token}`)
                .send({
                    fecha: '2025-12-01',
                    hora: '10:00',
                    nombreCliente: 'Test Client'
                });
            expect(res.statusCode).toEqual(201);
            expect(res.body).toHaveProperty('msg', 'Reserva creada');
        });
    });
});
\end{lstlisting}

\subsubsection{Pruebas de Carga (tests/k6\_load\_test.js)}
\begin{lstlisting}[language=JavaScript, caption=Script de carga para k6 con lógica de detección de entorno]
import http from 'k6/http';
import { check, sleep } from 'k6';

export const options = {
    stages: [
        { duration: '30s', target: 20 },
        { duration: '1m', target: 20 },
        { duration: '30s', target: 0 },
    ],
    thresholds: {
        http_req_duration: ['p(95)<500'],
    },
};

const HOST_IP = __ENV.K6_HOST || '127.0.0.1';
const BASE_URL = `http://${HOST_IP}:3000`;

export default function () {
    const payload = JSON.stringify({
        email: 'test@example.com',
        password: 'password123',
    });
    const params = { headers: { 'Content-Type': 'application/json' } };
    const res = http.post(`${BASE_URL}/api/auth/login`, payload, params);

    check(res, {
        'status is 200 or 400': (r) => r.status === 200 || r.status === 400,
    });
    sleep(1);
}
\end{lstlisting}

\subsubsection{Pruebas de Integración (tests/Reservas\_API.postman\_collection.json)} \label{sec:anexos_postman}
\begin{lstlisting}[language=JavaScript, caption=Selección de scripts de validación dinámica en Postman]
// Registro de Usuario
pm.test("Status code is 201", function () {
    pm.response.to.have.status(201);
});
pm.test("Mensaje de exito", function () {
    pm.expect(pm.response.json().msg).to.eql("Usuario creado");
});

// Login y Captura de Token
pm.test("Status code is 200", function () {
    pm.response.to.have.status(200);
});
pm.test("Token presente", function () {
    var jsonData = pm.response.json();
    pm.expect(jsonData).to.have.property('token');
    pm.environment.set("jwt_token", jsonData.token);
});

// Listar Reservas (Arreglo)
pm.test("Es un arreglo", function () {
    pm.expect(pm.response.json()).to.be.an('array');
});
\end{lstlisting}

\subsubsection{Plan de Pruebas de Estrés (tests/system\_test\_jmeter.jmx - XML)}
\begin{lstlisting}[language=XML, caption=Configuración del Thread Group y Sampler para Login en JMeter]
<ThreadGroup guiclass="ThreadGroupGui" testclass="ThreadGroup" testname="Grupo de Usuarios - Sistema">
  <stringProp name="ThreadGroup.num_threads">${__P(threads,50)}</stringProp>
  <stringProp name="ThreadGroup.ramp_time">${__P(rampup,10)}</stringProp>
  <elementProp name="ThreadGroup.main_controller" elementType="LoopController">
    <boolProp name="LoopController.continue_forever">false</boolProp>
    <stringProp name="LoopController.loops">1</stringProp>
  </elementProp>
</ThreadGroup>

<HTTPSamplerProxy guiclass="HttpTestSampleGui" testclass="HTTPSamplerProxy" testname="Login Request">
  <stringProp name="HTTPSampler.domain">localhost</stringProp>
  <stringProp name="HTTPSampler.port">3000</stringProp>
  <stringProp name="HTTPSampler.path">/api/auth/login</stringProp>
  <stringProp name="HTTPSampler.method">POST</stringProp>
  <boolProp name="HTTPSampler.postBodyRaw">true</boolProp>
  <elementProp name="HTTPsampler.Arguments" elementType="Arguments">
    <collectionProp name="Arguments.arguments">
      <elementProp name="" elementType="HTTPArgument">
        <stringProp name="Argument.value">{"email": "test@example.com", "password": "password123"}</stringProp>
      </elementProp>
    </collectionProp>
  </elementProp>
</HTTPSamplerProxy>
\end{lstlisting}

\subsubsection{Configuración de Integración Continua (.github/workflows/ci.yml)}
\begin{lstlisting}[language=XML, caption=Pipeline de CI/CD para automatización de pruebas y análisis en GitHub Actions]
name: Node.js CI

on:
  push:
    branches: [ "main" ]

jobs:
  build:
    runs-on: ubuntu-latest
    strategy:
      matrix:
        node-version: [18.x, 20.x]

    services:
      mongodb:
        image: mongo:latest
        ports:
          - 27017:27017
        options: >-
          --health-cmd "mongosh --eval 'db.runCommand({ping: 1})'"
          --health-interval 10s
          --health-timeout 5s
          --health-retries 5

    steps:
    - uses: actions/checkout@v4
    - name: Use Node.js ${{ matrix.node-version }}
      uses: actions/setup-node@v3
      with:
        node-version: ${{ matrix.node-version }}
        cache: 'npm'
    - name: Install dependencies
      run: npm install
    - name: Run tests
      run: npm test
      env:
        MONGO_URI: "mongodb://localhost:27017/test_db"
        JWT_SECRET: "test_secret_key"
    - name: SonarQube Scan
      uses: sonarsource/sonarqube-scan-action@master
      continue-on-error: true
      env:
        SONAR_TOKEN: ${{ secrets.SONAR_TOKEN }}
        SONAR_HOST_URL: ${{ secrets.SONAR_HOST_URL }}
\end{lstlisting}

\section{Referencias}
\begin{enumerate}
    \item \textbf{Repositorio del Proyecto:} \url{https://github.com/gamurigm/EvaluacionPractica1-reservas.git}
    \item OWASP ZAP Documentation - Web Security Testing Guide: \url{https://www.zaproxy.org/docs/}
    \item Apache JMeter Documentation - Performance and Load Testing: \url{https://jmeter.apache.org/usermanual/}
\end{enumerate}
